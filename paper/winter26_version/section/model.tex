\section{Model Selection} \label{sec:model}

    % In this section, we evaluate the out-of-sample performance of different models in predicting firm-year-level emissions. For each model, we conduct a Leave-One-Out Cross-Validation (LOOCV) using firms for which we obseve yearly emissions and compute the nRMSE. The exercise amounts to doing repeated out-of-sample predictions. The idea is to randomly pick a firm among the ones for which we observe emissions, exclude it from the training data, fit the model using the remaining firms, use the fitted model to predict yearly emissions for all years for the firm that was initially excluded, and then use the fact that we observe this firm's emissions to assess how far the inference based on our model is to the actual observed value. See Algorithm \ref{algorithm: loocv} for a detailed description.
    
    % We start off with following model as benchmark:

    % log(emissions_it) = log(revenue_it) + year FE + sector FE

    % We introduce four ingredients to model (\ref{}) and assess the change in out-of-sample predictive performance for each of them in turn. They are:

    % i. flexiblity over the level at which we define the sector fixed effects.
    % ii. introduce the fuel consumption proxy.
    % iii. a intermediary to identify pollutants.
    % iv. calibrate the firm-level predictions to match sector-level known aggregates.

    % Discuss why sector-level at a granular level (5-digit) is useful but problematic.

    % Show the following:
        % Distribution of sector support in the training sample: how many sector FE are estimated with very few firm-years

        % "Unrestricted sector fixed effects may be estimated with substantial noise"?

        % “What fraction of firms (or emissions) would rely on sector FE with weak or no support at deployment?”

    % Introduce partial pooling: formal and intuition.

    % Discuss results

    % Introduce all the tweaks to the fuel consumption proxy.

    % Explain we ran the model for all possible combinations of the tweaks and we chose the combination that provided the best nRMSE within the LOOCV.

    % Discuss the results (what is the chosen proxy? what's the number of surviving obs? is this the proxy that also delivers the best nRMSE after steps 1,3?)

    % Introduce step 1. Discuss results. What's the threshold we chose? Is is based on the same fuel consumption proxy as the one chosen in the step above?

    % Introduce step 3. Discuss results. 

    % Introduce LOSOCV. Discuss results. 

    % Table with four columns: nRMSE, MAPD (E_i > 0), FPR (TPR?), Spearman rho

    % And 6 rows: Benchmark, Partial pooling, Fuel consumption proxy, Weight by prob(E_i > 0), Match sectoral aggregates, LOSOCV
    


    

    
    