    \section{A Proxy for Fuel Consumption}

    This section motivates our proxy for fossil fuel consumption, formally defines it, and presents suggestive evidence of its information content.

    \subsection{Motivation and Definition}

    Emissions of greenhouse gases (GHG) from stationary industrial installations originate from two broad sources: industrial processes and the combustion of fossil fuels. Emissions from industrial processes depend on the quantities and composition of specific material inputs, whereas emissions from fuel combustion depend on the quantities and characteristics of fossil fuels used.
    
    In line with its reporting obligations under international climate agreements, Belgium annually publishes a National Inventory of GHG Emissions, which provides a consistent and transparent accounting of emissions across sectors over time. In recent years, this inventory reports, for a subset of economic activities, the share of emissions that is regulated under the EU ETS. We report these figures in Table \complete. Virtually all emissions originating from industrial processes are regulated by the EU ETS.\footnote{Activities under the second heading refer to emissions from industrial processes. They are: ``2A Mineral products", ``2B Chemical industry", ``2C Metal production", ``2D3 Non-energy products from fuels and solvent use", and ``2H Other". Activity 2H mainly includes the production of pulp and paper. For all of them except for 2D3, for which the total emissions are negligibly small, EU ETS regulates 100\% of the installations.}  Consequently, when inferring emissions for non-EU ETS firms in Belgium, we restrict attention to emissions from the combustion of fossil fuels.
    
    Combustion-related emissions are proportional to the quantities of each fuel consumed, weighted by fuel-specific emission factors. These emission factors depend on the carbon content and oxidation characteristics of each fuel and may vary over time and across fuel types, even within narrowly defined fuel categories. In practice, however, it is common to apply homogeneous emission factors within sufficiently granular fuel categories, an approach widely adopted in the literature \citep{shapiro2018, colmeretal2023}. Thus, for any given firm $i$ in year $t$ emissions from the combustion of fossil fuels are given by

    $$ E_{it} = \sum_{f \in \text{fossil fuels}} \text{EF}_{f} \times q_{ift}$$

    \noindent where EF$_{f}$ is the emission factor of fuel $f$ in year $t$ and $q_{ift}$ is the amount firm $i$ consumes of fuel $f$ in year $t$.

    Unfortunately, we do not have access to data on the quantity of fuel combusted by firm in Belgium. We thus propose to build a proxy for it using Customs and firm-to-firm transactions data.

    We first establish the fact that fuel supply in Belgium comes from imports. Figure \ref{fig: total supply vs imports} plots total supply (imports + domestic production) against imports across all fossil fuels. Each observation is a fossil fuel-year pair. Across all fuels and years, imports represent the vast majority of the quantity supplied - it tracks almost perfectly the 45-degree line.

    \begin{figure}
        \centering
        \caption{Total Supply vs Imports Across Fossil Fuels and Years}
        \includegraphics[scale = 0.5]{dec25_version/figures/total_supply_vs_imports.png}
        \label{fig: total supply vs imports}
        \caption*{\footnotesize \textit{Notes}: Figure plots aggregate data on total supply against imports by fossil fuel and year in Belgium. The variables come from the Complete Energy Balances available in Eurostat (data code \textit{nrg\_bal\_c}). Total supply is the sum of Primary production (data code \textit{PPRD}) and imports (data code \textit{IMP}). Both are measured in Terajoules (TJ). Each blue dot is a fuel $\times$ year and the dotted black line is the 45-degree line. We include all years between 2005 and 2022. In the data, fuels are defined at the level of the Standard International Energy Product Classification (SIEC). We include all fuels except the ones not used to generate GHGs in stationary installations and umbrella categories, to avoid double-counting. The following SIEC codes are excluded: E7000, H8000, FE, C0000X0350-0370, C0350-0370, P1000, O4000XBIO, R5110-5150\_W6000RI, R5210B, R5210P, R5220B, R5220P, R5230P, R5230B, R5290, R5300.}
    \end{figure}

    This motivates us to define our proxy for fuel consumption at the firm-year-level as the total amount any given firm purchases from fuel importers in any given year - purchases from fuel importers are probably strongly correlated with fossil fuel consumption and, to the extent that all fuels are imported into Belgium, it should contain the vast majority of fuel consumption by Belgian firms. 

    Formally, for firm $i$ in year $t$ we define

    \begin{equation}
        \label{eq: fuel consumption}
        fuel\_consumption_{it} := \sum_{j \in \text{fuel importers}} x_{ijt}
    \end{equation}

    We identify fuel importers using customs data. If in any given year a firm imports into Belgium a positive amount of fossil fuels, we flag it as a fuel importer. We start off with a broad definition of fuels: all products identified by CN 8-digit codes in chapter 27 of the World Harmonized System (HS).

    For fuel importers, their fuel consumption proxy is

    \begin{equation}
        \label{eq: fuel consumption}
        fuel\_consumption_{it} := \sum_{j \in \text{fuel importers}} x_{ijt} + \sum_{f \in \text{fossil fuels}} import_{ift}
    \end{equation}

    \noindent that is, the sum of total purchases from fuel importers plus their own imports of fossil fuels. 

    \noindent \textbf{Variants.} There are reasons to be skeptical about the information content of our proxy. For example, fuel importers can consume the fuels themselves. In that case, purchases from fuel importers would capture purchases of goods that embed emissions, but not purchases of fuels. Alternatively, even if fuel importers do mainly act as  fuel suppliers, downstream consumers might re-sell fuel further downstream as opposed to burning the fuels themselves. We suggest some variants of our baseline proxy that address these and other potential sources of noise. Ultimately, which proxy contains the smallest noise-to-signal ratio will be determined empirically by the amount of variance in emissions across observations it is able to explain out-of-sample.

    We consider the following variants to the fuel consumption proxy:

    \begin{itemize}
        \item[1.] \textbf{Emission-intensity-weighted purchases.} We construct emission-intensity-weighted imports by importer and year as the sum, across fuels, of imported quantities weighted by fuel-specific emission factors. Customs data provide firm–product–year import quantities reported in kilograms, while emission factors are expressed in tons of CO$_2$ per terajoule. To reconcile these units, we convert fuel quantities from mass to energy content using net calorific values (NCVs).

        This procedure involves two caveats. First, import quantities reported in Customs data are not always reliable. Second, NCVs are obtained from Eurostat, which classifies fuels according to the SIEC. In Annex \complete, we describe in detail our procedures for cleaning firm–fuel–year quantities and for constructing a CN 8-digit–to-SIEC crosswalk.

        We then define the importer-year emission-intensity of fuel imports as
        
        \begin{equation*}
            EI_{jt} = \frac{1}{\sum_{f \in \text{fuels}} x_{jft}} \sum_{f \in \text{fuels}} q_{jft} \times NCV_{ft} \times EF_{ft} 
        \end{equation*}

        \noindent where $x_{jft}$ is the euro amount firm $j$ imported of fuel $f$ in year $t$. The emission-intensity-weighted variant of our fuel consumption is then defined as

        \begin{equation}
            fuel\_consumption^{EI}_{it} :=  \sum_{j \in \text{fuel importers}} x_{ijt} \times EI_{jt}
        \end{equation}

        \item[2.] \textbf{Fuels used in GHG-emitting activities in stationary installations.} Variants \textbf{2a.}, \textbf{2b.} and \textbf{2c.} are defined the same way as (\ref{eq: fuel consumption}), but the set of fuel importers is different. In each of them, we restrict our definition of fossil fuels in Customs and reclassify firms as fuel importers accordingly.

        \vspace{0.3cm}
        
        \begin{itemize}

            \item[2a.] \textbf{Following Energy Balances.} From Energy Balances, we obtain the list of fossil fuels used in Belgium in GHG-emitting activities: \textit{Final consumption - energy use; Transformation input - energy use; and Energy sector - energy use}. We identify the fuels consumed by at least one of these three categories in at least one year between 2005 and 2022, and redefine fuel importers as firms that import positive amounts of one of the fuels identified by energy balances as used in Belgium for GHG-emitting activities.

            \vspace{0.3cm}

            \item[2b.] \textbf{Sector-specific fuels.} Energy Balances report fuel consumption in GHG-emitting activities, broken down by fuel type and economic activity. Examples of such categories include \textit{Final consumption – industry sector – iron and steel – energy use}, \textit{Transformation input – coke ovens – energy use, and Energy sector – blast furnaces – energy use}. We map each Energy Balance category to a NACE 2-digit sector (the full mapping is provided in Appendix \complete). Using this mapping, we construct buyer-specific sets of fuel importers. Specifically, for each buyer, we use its NACE 2-digit code—obtained from Annual Accounts—to identify the set of fuels used in GHG-emitting activities by that sector, and we restrict attention to firms that import those fuels.

            \vspace{0.3cm}

            \item[2c.] \textbf{Following GPT-5.2.} We ask GPT-5.2, the model that powered ChatGPT in February 2026, to identify which CN 8-digit codes from chapter 27 of the HS refer to fossil fuels used in GHG-emitting activities in industrial stationary installations. We then restrict our fossil fuel definition to this subset of fuels and reclassify fuel importers as the set of firm-year observations that import any positive amount of this restricted list of fuels. Appendix \complete provides the list of all goods included in chapter 27 of the HS and identifies the ones flagged by GPT-5.2. We also report the specific prompt we used in Appendix \complete. 

        \end{itemize}
            
        \vspace{0.3cm}
        
        \item[3.] \textbf{Importers that are likely to be fuel suppliers.} We use importers’ NACE 2-digit codes to classify them according to their likelihood of being fossil fuel suppliers. We define three categories: \textbf{highly likely}, \textbf{medium likely}, and \textbf{not likely}. Appendix \complete reports the classification of NACE 2-digit sectors into these three groups. Using this classification, we define two variants of the fuel consumption proxy:

        \begin{align*}
        fuel\_consumption_{it}^{high}     &:= \sum_{j \in \text{high}} x_{ijt} \\
        fuel\_consumption_{it}^{high\_med} &:= \sum_{j \in \text{high} \cup \text{medium}} x_{ijt}
        \end{align*}

        \noindent that is, $fuel\_consumption_{it}^{high}$ restricts the set of fuel importers to firms whose NACE 2-digit sector indicates a high likelihood of supplying fossil fuels, while $fuel\_consumption_{it}^{high}$ additionally includes firms whose sector indicates a medium likelihood.

        \vspace{0.3cm}

        \item[4.] \textbf{Exclude EU ETS-regulated fuel importers.} Because high-emitting fuel importers are more likely to use imported fuels in their own production rather than sell them downstream, we construct a final classification of fuel importers that excludes firms covered by the EU ETS:

        \begin{equation}
            fuel\_consumption_{it}^{non-EUETS} := \sum_{j \in \text{fuel importers}/\text{EUETS}} x_{ijt}
        \end{equation}
    \end{itemize}

    In Table \complete we present the number of firms classified as fuel importers and the euros amount (expressed in 2005 values) imported of fuels in 2005, 2010, 2015, and 2020 for each of the variants. 



