\section{Data} \label{sec:data}

     \noindent \textbf{Firm Characteristics.} We obtain firm-year-level revenue for all Belgian private firms from the National Bank of Belgium (NBB)'s Annual Accounts. This is the data set used to construct the National Income and Product Accounts in Belgium.\footnote{See p. 81 in \url{https://www.nbb.be/doc/dq/e_method/gni_methodological_inventory_belgium_version_2022_publication.pdf} for further details.} The data also contains firms' NACE 5-digit sector codes.
     
    \noindent \textbf{Firm-to-firm Transactions.} The NBB's business-to-business (B2B) transactions database covers almost all bilateral business transactions between Belgian firms in the period 2002-2022. It is a virtually complete network representation of the private sector in Belgium \citep{duprez2023belgian}.
    
    The data contains firm-level unique identifiers for all Belgian private firms that have taken part in at least one transaction with other Belgian firms in the period 2002-2022.
    
    For each year and each buyer-supplier relationship between any two Belgian private firms, it informs the yearly monetary transaction value in nominal euros. We do not observe which products were transacted or their quantity.
    
    \noindent \textbf{Fuel Imports and Domestic Supply.} We obtain the universe of firm-product-year-level imports and exports by Belgian firms from Customs.  For each Belgian firm, it reports the amount, in values and quantities, of imports and exports by destination and source country for each product classified at the 8-digit combined nomenclature (CN), with around 10,000 distinct products.

    For each firm-year, we collect imports of products with CN codes in chapter 27 of the World Harmonized System (HS), which includes all fossil fuels, among other products.

    Additionally, from Eurostat we collect aggregate data on yearly imports and total supply (imports + domestic production) of fossil fuels in Belgium. 

    \noindent \textbf{Sector-level Fuel Consumption.} From energy balances, we collect the fuels used by each sector-year in Belgium in GHG-emitting activities. The data is available in Eurostat and in Table \complete we list the entries from the energy balances we classify as GHG-emitting activities. Fuels are classified according to the Standard International Energy Product Classification (SIEC). 

    \noindent \textbf{Firm-level Emissions.}The European Union Transaction Log (EUTL) is the main reporting and monitoring tool of the European Union Emissions Trading System (EUETS), the world's first and largest market-based carbon pricing initiative and the flagship carbon pricing policy in Europe.
    
    The EUETS' main purpose is to establish a price for the right to emit greenhouse gases (GHGs). Instead of directly setting the price, each year the EU Commission imposes a cap on the continent-wide aggregate emissions from power and manufacturing plants across the 31 participating countries. 

    Within the cap, the EU distributes emission allowances. One allowance gives its holder the right to emit one ton of carbon dioxide-equivalent (CO$_2$-eq) GHGs. Throughout the year, firms can trade allowances in a permit market, which determines their price. Scarce allowances command a positive price in the permit market. Every year, regulated installations need to surrender allowances to the regulatory authority at least equal to their verified emissions of the previous year. 

    The EUTL makes publicly available yearly information on the amount of emission allowances surrendered by each regulated installation. Each installation has a unique firm-level identifier, but a firm might own multiple installations. We define a firm's emissions in any given year to be equal to the sum of emissions in that year across all installations it owns. This gives us emissions for all Belgian firms regulated by the EUETS for all years between 2005 and 2022.

    \noindent \textbf{Sector-level Emissions.} National inventories provide aggregate emissions by sector and year, from 2005 to 2022. Economic activities are grouped into Common Reporting Format (CRF) categories in National Inventories. The CRF categories map into NACE 2-digit codes according to \complete. Additionally, Annex XII to the National Inventories for the years 2024 and 2025 compares aggregate emissions with the volume of emissions covered by the EUETS by category of economic activity, for selected categories.

    \noindent \textbf{Emission Factors and Calorific Values.} We obtain emission factors by fuel from the 2006 IPCC Guidelines for National Greenhouse Gas Inventories [CITE\complete]. We use the default emission factors in Table 2.2, Vol. 2, Chapter 2. Fuels are categorized into IPCC-specific groups - it does not follow the SIEC or CN classifications. 
    
    We compile fuel-level Net Calorific Values (NCV) from two different sources. Eurostat provides NCV for most fuels in Belgium. Whenever available, we use the entry \textit{Net calorific value - average} for each fuel. If not available, we complement this data with default NCVs by fuel from Table 4.1 in the International Recommendations for Energy Statistics (IRES) [CITE\complete]. In both Eurostat and the IRES fuels are classified according to SIEC. 

    \noindent \textbf{Fuel Concordance.} There is no ready-to-use or officially harmonized crosswalk linking product-level trade classifications from the Combined Nomenclature (CN) to fuel categories used in the IPCC emissions accounting framework or to SIEC. Each system was designed for a distinct purpose—customs and trade statistics (CN), emissions inventories (IPCC), and energy balances (SIEC)—and they differ in both their level of aggregation and their underlying conceptual definitions. As a result, existing concordances are partial, inconsistent, or tailored to specific applications, and cannot be directly applied to merge the datasets used in this study. To integrate them, we therefore constructed the necessary crosswalks manually, combining technical descriptions of fuels and product characteristics to map CN codes to IPCC fuel types and SIEC energy products in a consistent manner. In Table \complete we provide examples. 