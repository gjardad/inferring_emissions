\documentclass[11pt,  reqno]{amsart}
%\documentclass[11pt]{amsart}
%\usepackage{harvard}
\usepackage{graphicx,amsmath,amssymb,epsfig,epstopdf}
\usepackage{subcaption}
\usepackage{color}
\newcommand{\complete}{\textcolor{red}{XX} }
\usepackage{natbib}
\usepackage{url} 
\usepackage{hyperref}
\hypersetup{
    colorlinks=true,
    citecolor=blue,
    linkcolor=blue,
    urlcolor=blue
    }
\usepackage{float} % Load the float package
\usepackage{tikz}
\usetikzlibrary{positioning}
\usetikzlibrary{calc}
\usepackage{comment}
\usepackage{booktabs}
\usepackage{lscape}
\usepackage[gen]{eurosym}
\usepackage[ruled,vlined]{algorithm2e}
\usepackage{longtable}


\tikzset{place/.style={circle,thick,draw=blue!75,fill=blue!20,minimum size=10mm}}
\tikzset{place2/.style={circle,thick,draw=blue!75,fill=green!20,minimum size=10mm}}
\tikzset{place3/.style={circle,thick,draw=red!75,fill=orange!20,minimum size=10mm}}
\tikzset{main node/.style={circle,draw,font=\sffamily\small\bfseries,minimum size=10mm}}

%\bibliographystyle{econometrica}
%\usepackage{showkeys}
\long\def\comment#1{}
\oddsidemargin +0.2in
 \evensidemargin +0.2in
\topmargin 40pt \textheight 8.1in \textwidth 6in
\linespread{1.3}\parskip .05in
%% ----------------------------------------------------------------
\vfuzz2pt % Don't report over-full v-boxes if over-edge is small
\hfuzz2pt % Don't report over-full h-boxes if over-edge is small
% THEOREMS -------------------------------------------------------
\long\def\comment#1{}
\newtheorem{theorem}{Theorem}
\newtheorem{conjecture}{Conjecture}
\newtheorem{corollary}{Corollary}
\newtheorem{lemma}{Lemma}
\newtheorem{proposition}{Proposition}
\newtheorem{assumption}{Assumption}
\theoremstyle{definition}
\newtheorem{definition}{Definition}
\newtheorem{step}{Step}
\newtheorem{remark}{Comment}
\newtheorem{claim}{Claim}
%\numberwithin{remark}{section}
\newtheorem{example}{Example}
%\numberwithin{equation}{section}

\newcommand{\citen}{\citeasnoun}


\newcommand{\norm}[1]{\left\Vert#1\right\Vert}
\newcommand{\abs}[1]{\left\vert#1\right\vert}
\newcommand{\set}[1]{\left\{#1\right\}}
\newcommand{\Real}{\mathbb R}
\newcommand{\eps}{\varepsilon}
\newcommand{\To}{\longrightarrow}
\newcommand{\BX}{\mathbf{B}(X)}
\newcommand{\A}{\mathcal{A}}
\newcommand{\what}{\widehat}
\newcommand{\eg}{e.g., \xspace}
\newcommand{\ie}{i.e.,\xspace}
\newcommand{\etc}{etc.\@\xspace}
\newcommand{\iid}{\emph{i.i.d.}\ }
\newcommand{\etal}{et.\ al.\ }
\newcommand{\PP}{P}
\newcommand{\be}{\begin{eqnarray}}
\newcommand{\ee}{\end{eqnarray}}
\newcommand{\Tau}{\mathcal{T}}
\newcommand{\EE}{E}
\newcommand{\Ss}{\mathcal{S}}
\newcommand{\GG}{\Gamma}
\newcommand{\NN}{\mathbf{N}}
\newcommand{\JJ}{J}
\newcommand{\XX}{\mathcal{X}}
\newcommand{\EC}{\mathcal{E}}
\newcommand{\U}{\mathcal{U}}
\newcommand{\asex}{ \text{ as $ \tau T  \rightarrow k$  and $\tau \searrow 0$ }}
\newcommand{\asint}{ \text{ as $ \tau T  \rightarrow \infty$  and $ \tau \searrow  0$ }}

\newcommand{\ind}{ \overset{d}{\longrightarrow}}
\newcommand{\inp}{ \overset{p}{\longrightarrow}}
\newcommand{\ba}{\begin{array}}
\newcommand{\ea}{\end{array}}
\newcommand{\bs}{\begin{align}\begin{split}\nonumber}
\newcommand{\bsnumber}{\begin{align}\begin{split}}
\newcommand{\es}{\end{split}\end{align}}
\newcommand{\doublespace}{\linespread{1.1}}
\newcommand{\singlespace}{\linespread{0.85}}
\newcommand{\fns}{\singlespace\footnotesize}
\newcommand{\cooldate}{\begin{flushright}\footnotesize\today\normalsize\end{flushright}}
\newcommand{\expect}{\mathcal{E}}
\newcommand{\sss}{\scriptscriptstyle}
\newcommand{\n}{n}
\newcommand{\inpr}{ \overset{p^*_{\scriptscriptstyle n}}{\longrightarrow}}
\newcommand{\T}{ \scriptscriptstyle T}
\newcommand{\LL}{ \mathcal{L}}
\newcommand{\M}{ \scriptscriptstyle M}
\newcommand{\al}{ \scriptscriptstyle A}
\renewcommand{\(}{\left(}
\renewcommand{\)}{\right)}
\renewcommand{\[}{\left[}
\renewcommand{\]}{\right]}
\renewcommand{\hat}{\widehat}

\newcommand{\kk}{\kappa}
\newcommand{\wpco}{\text{ w. pr. $\geq 1 - \epsilon$  } }
\newcommand{\jb}{ { {\mathbf{j}}} }
\newcommand{\BB}{\Bbb{B}}
\newcommand{\inlaw}{\overset{D}\sim}
\newcommand{\QED}{$\blacksquare$}
\newcommand{\aT}{A_T}
\newcommand{\kb}{\text{\fns{$\frac{k}{b}$}}}
\newcommand{\kT}{\text{\fns{$\frac{k}{T}$}}}
\renewcommand{\theequation}{\thesection.\arabic{equation}}
\renewcommand{\fns}{\footnotesize}
\newcommand{\smalls}{\fontsize}
\newcommand{\cc}{\mathbf{c}}
\newcommand{\B}{\mathcal{B}}
\newcommand{\F}{\mathcal{F}}
\newcommand{\G}{\mathcal{G}}
\newcommand{\C}{\mathcal{C}}
\newcommand{\K}{\mathcal{K}}
\newcommand{\N}{\mathcal{N}}
\newcommand{\W}{\mathcal{W}}
\newcommand{\R}{\mathcal{R}}
\newcommand{\Z}{\mathcal{Z}}
\newcommand{\QQ}{\widetilde{Q}}
\newcommand{\nnocr}{\nonumber\\}
\newcommand{\nno}{\nonumber}
\newcommand{\Gn}{\mathbb{G}_n}
\newcommand{\Gp}{\mathbf{G}}
\newcommand{\Pn}{\mathbb{P}_n}
\newcommand{\Pp}{P}
\newcommand{\En}{\mathbb{E}_n}
\newcommand{\Ep}{E}
\newcommand{\barf}{\overline{f}}
\newcommand{\barfp}{\overline{f'}}
\newcommand{\underf}{\underline{f}}
\newcommand{\Uniform}{{\text{Uniform}}}
\newcommand{\trace}{{\text{trace}}}
\newcommand{\ltwob}
{\tilde{\beta}^{^*}}
%\def\RR{{\rm I\kern-0.18em R}}
\def\RR{ {\Bbb{R}}}
\def\mK{{\mathcal K}}
\def\hn{{h}}

\begin{document}

\title[]{Inferring Firms' Emissions Using Transactions Data}

\author[]{Gabriel Jardanosvki\textsuperscript{*} \and Gert Bijnens\textsuperscript{$\dagger$}}
\thanks{\noindent \textsuperscript{*}Jardanovski (corresponding author): Northwestern University. E-mail: \texttt{gabrielj@u.northwestern.edu}.}
\thanks{\textsuperscript{$\dagger$} Bijnens: National Bank of Belgium. E-mail: \texttt{gert.bijnens@nbb.be}.}

\begin{abstract}
This note proposes a procedure to infer firms' yearly emissions using business-to-business (B2B) transaction data and balance-sheet data from Belgium. We propose a firm-year-level proxy for the consumption of fossil fuels and formally evaluate its performance as a predictor of emissions using a sample of Belgian firms for which we observe yearly emissions. We find that the proxy explains a meaningful share of the variance of emissions across firm-years even after controlling for firm size. In our preferred specification, the out-of-sample R$^2$ is 0.42. We also discuss future strategies to improve the inference procedure. 
\end{abstract}

\maketitle

\pagebreak

\include{winter26_version/section/intro}

\section{Data} \label{sec:data}

     \noindent \textbf{Firm Characteristics.} We obtain firm-year-level revenue for all Belgian private firms from the National Bank of Belgium (NBB)'s Annual Accounts. This is the data set used to construct the National Income and Product Accounts in Belgium.\footnote{See p. 81 in \url{https://www.nbb.be/doc/dq/e_method/gni_methodological_inventory_belgium_version_2022_publication.pdf} for further details.} The data also contains firms' NACE 5-digit sector codes.
     
    \noindent \textbf{Firm-to-firm Transactions.} The NBB's business-to-business (B2B) transactions database covers almost all bilateral business transactions between Belgian firms in the period 2002-2022. It is a virtually complete network representation of the private sector in Belgium \citep{duprez2023belgian}.
    
    The data contains firm-level unique identifiers for all Belgian private firms that have taken part in at least one transaction with other Belgian firms in the period 2002-2022.
    
    For each year and each buyer-supplier relationship between any two Belgian private firms, it informs the yearly monetary transaction value in nominal euros. We do not observe which products were transacted or their quantity.
    
    \noindent \textbf{Fuel Imports and Domestic Supply.} We obtain the universe of firm-product-year-level imports and exports by Belgian firms from Customs.  For each Belgian firm, it reports the amount, in values and quantities, of imports and exports by destination and source country for each product classified at the 8-digit combined nomenclature (CN), with around 10,000 distinct products.

    For each firm-year, we collect imports of products with CN codes in chapter 27 of the World Harmonized System (HS), which includes all fossil fuels, among other products.

    Additionally, from Eurostat we collect aggregate data on yearly imports and total supply (imports + domestic production) of fossil fuels in Belgium. 

    \noindent \textbf{Sector-level Fuel Consumption.} From energy balances, we collect the fuels used by each sector-year in Belgium in GHG-emitting activities. The data is available in Eurostat and in Table \complete we list the entries from the energy balances we classify as GHG-emitting activities. Fuels are classified according to the Standard International Energy Product Classification (SIEC). 

    \noindent \textbf{Firm-level Emissions.}The European Union Transaction Log (EUTL) is the main reporting and monitoring tool of the European Union Emissions Trading System (EUETS), the world's first and largest market-based carbon pricing initiative and the flagship carbon pricing policy in Europe.
    
    The EUETS' main purpose is to establish a price for the right to emit greenhouse gases (GHGs). Instead of directly setting the price, each year the EU Commission imposes a cap on the continent-wide aggregate emissions from power and manufacturing plants across the 31 participating countries. 

    Within the cap, the EU distributes emission allowances. One allowance gives its holder the right to emit one ton of carbon dioxide-equivalent (CO$_2$-eq) GHGs. Throughout the year, firms can trade allowances in a permit market, which determines their price. Scarce allowances command a positive price in the permit market. Every year, regulated installations need to surrender allowances to the regulatory authority at least equal to their verified emissions of the previous year. 

    The EUTL makes publicly available yearly information on the amount of emission allowances surrendered by each regulated installation. Each installation has a unique firm-level identifier, but a firm might own multiple installations. We define a firm's emissions in any given year to be equal to the sum of emissions in that year across all installations it owns. This gives us emissions for all Belgian firms regulated by the EUETS for all years between 2005 and 2022.

    \noindent \textbf{Sector-level Emissions.} National inventories provide aggregate emissions by sector and year, from 2005 to 2022. Economic activities are grouped into Common Reporting Format (CRF) categories in National Inventories. The CRF categories map into NACE 2-digit codes according to \complete. Additionally, Annex XII to the National Inventories for the years 2024 and 2025 compares aggregate emissions with the volume of emissions covered by the EUETS by category of economic activity, for selected categories.

    \noindent \textbf{Emission Factors and Calorific Values.} We obtain emission factors by fuel from the 2006 IPCC Guidelines for National Greenhouse Gas Inventories [CITE\complete]. We use the default emission factors in Table 2.2, Vol. 2, Chapter 2. Fuels are categorized into IPCC-specific groups - it does not follow the SIEC or CN classifications. 
    
    We compile fuel-level Net Calorific Values (NCV) from two different sources. Eurostat provides NCV for most fuels in Belgium. Whenever available, we use the entry \textit{Net calorific value - average} for each fuel. If not available, we complement this data with default NCVs by fuel from Table 4.1 in the International Recommendations for Energy Statistics (IRES) [CITE\complete]. In both Eurostat and the IRES fuels are classified according to SIEC. 

    \noindent \textbf{Fuel Concordance.} There is no ready-to-use or officially harmonized crosswalk linking product-level trade classifications from the Combined Nomenclature (CN) to fuel categories used in the IPCC emissions accounting framework or to SIEC. Each system was designed for a distinct purpose—customs and trade statistics (CN), emissions inventories (IPCC), and energy balances (SIEC)—and they differ in both their level of aggregation and their underlying conceptual definitions. As a result, existing concordances are partial, inconsistent, or tailored to specific applications, and cannot be directly applied to merge the datasets used in this study. To integrate them, we therefore constructed the necessary crosswalks manually, combining technical descriptions of fuels and product characteristics to map CN codes to IPCC fuel types and SIEC energy products in a consistent manner. In Table \complete we provide examples. 

    \section{A Proxy for Fuel Consumption}

    This section motivates our proxy for fossil fuel consumption, formally defines it, and presents suggestive evidence of its information content.

    \subsection{Motivation and Definition}

    Emissions of greenhouse gases (GHG) from stationary industrial installations originate from two broad sources: industrial processes and the combustion of fossil fuels. Emissions from industrial processes depend on the quantities and composition of specific material inputs, whereas emissions from fuel combustion depend on the quantities and characteristics of fossil fuels used.
    
    In line with its reporting obligations under international climate agreements, Belgium annually publishes a National Inventory of GHG Emissions, which provides a consistent and transparent accounting of emissions across sectors over time. In recent years, this inventory reports, for a subset of economic activities, the share of emissions that is regulated under the EU ETS. We report these figures in Table \complete. Virtually all emissions originating from industrial processes are regulated by the EU ETS.\footnote{Activities under the second heading refer to emissions from industrial processes. They are: ``2A Mineral products", ``2B Chemical industry", ``2C Metal production", ``2D3 Non-energy products from fuels and solvent use", and ``2H Other". Activity 2H mainly includes the production of pulp and paper. For all of them except for 2D3, for which the total emissions are negligibly small, EU ETS regulates 100\% of the installations.}  Consequently, when inferring emissions for non-EU ETS firms in Belgium, we restrict attention to emissions from the combustion of fossil fuels.
    
    Combustion-related emissions are proportional to the quantities of each fuel consumed, weighted by fuel-specific emission factors. These emission factors depend on the carbon content and oxidation characteristics of each fuel and may vary over time and across fuel types, even within narrowly defined fuel categories. In practice, however, it is common to apply homogeneous emission factors within sufficiently granular fuel categories, an approach widely adopted in the literature \citep{shapiro2018, colmeretal2023}. Thus, for any given firm $i$ in year $t$ emissions from the combustion of fossil fuels are given by

    $$ E_{it} = \sum_{f \in \text{fossil fuels}} \text{EF}_{f} \times q_{ift}$$

    \noindent where EF$_{f}$ is the emission factor of fuel $f$ in year $t$ and $q_{ift}$ is the amount firm $i$ consumes of fuel $f$ in year $t$.

    Unfortunately, we do not have access to data on the quantity of fuel combusted by firm in Belgium. We thus propose to build a proxy for it using Customs and firm-to-firm transactions data.

    We first establish the fact that fuel supply in Belgium comes from imports. Figure \ref{fig: total supply vs imports} plots total supply (imports + domestic production) against imports across all fossil fuels. Each observation is a fossil fuel-year pair. Across all fuels and years, imports represent the vast majority of the quantity supplied - it tracks almost perfectly the 45-degree line.

    \begin{figure}
        \centering
        \caption{Total Supply vs Imports Across Fossil Fuels and Years}
        \includegraphics[scale = 0.5]{dec25_version/figures/total_supply_vs_imports.png}
        \label{fig: total supply vs imports}
        \caption*{\footnotesize \textit{Notes}: Figure plots aggregate data on total supply against imports by fossil fuel and year in Belgium. The variables come from the Complete Energy Balances available in Eurostat (data code \textit{nrg\_bal\_c}). Total supply is the sum of Primary production (data code \textit{PPRD}) and imports (data code \textit{IMP}). Both are measured in Terajoules (TJ). Each blue dot is a fuel $\times$ year and the dotted black line is the 45-degree line. We include all years between 2005 and 2022. In the data, fuels are defined at the level of the Standard International Energy Product Classification (SIEC). We include all fuels except the ones not used to generate GHGs in stationary installations and umbrella categories, to avoid double-counting. The following SIEC codes are excluded: E7000, H8000, FE, C0000X0350-0370, C0350-0370, P1000, O4000XBIO, R5110-5150\_W6000RI, R5210B, R5210P, R5220B, R5220P, R5230P, R5230B, R5290, R5300.}
    \end{figure}

    This motivates us to define our proxy for fuel consumption at the firm-year-level as the total amount any given firm purchases from fuel importers in any given year - purchases from fuel importers are probably strongly correlated with fossil fuel consumption and, to the extent that all fuels are imported into Belgium, it should contain the vast majority of fuel consumption by Belgian firms. 

    Formally, for firm $i$ in year $t$ we define

    \begin{equation}
        \label{eq: fuel consumption}
        fuel\_consumption_{it} := \sum_{j \in \text{fuel importers}} x_{ijt}
    \end{equation}

    We identify fuel importers using customs data. If in any given year a firm imports into Belgium a positive amount of fossil fuels, we flag it as a fuel importer. We start off with a broad definition of fuels: all products identified by CN 8-digit codes in chapter 27 of the World Harmonized System (HS).

    For fuel importers, their fuel consumption proxy is

    \begin{equation}
        \label{eq: fuel consumption}
        fuel\_consumption_{it} := \sum_{j \in \text{fuel importers}} x_{ijt} + \sum_{f \in \text{fossil fuels}} import_{ift}
    \end{equation}

    \noindent that is, the sum of total purchases from fuel importers plus their own imports of fossil fuels. 

    \noindent \textbf{Variants.} There are reasons to be skeptical about the information content of our proxy. For example, fuel importers can consume the fuels themselves. In that case, purchases from fuel importers would capture purchases of goods that embed emissions, but not purchases of fuels. Alternatively, even if fuel importers do mainly act as  fuel suppliers, downstream consumers might re-sell fuel further downstream as opposed to burning the fuels themselves. We suggest some variants of our baseline proxy that address these and other potential sources of noise. Ultimately, which proxy contains the smallest noise-to-signal ratio will be determined empirically by the amount of variance in emissions across observations it is able to explain out-of-sample.

    We consider the following variants to the fuel consumption proxy:

    \begin{itemize}
        \item[1.] \textbf{Emission-intensity-weighted purchases.} We construct emission-intensity-weighted imports by importer and year as the sum, across fuels, of imported quantities weighted by fuel-specific emission factors. Customs data provide firm–product–year import quantities reported in kilograms, while emission factors are expressed in tons of CO$_2$ per terajoule. To reconcile these units, we convert fuel quantities from mass to energy content using net calorific values (NCVs).

        This procedure involves two caveats. First, import quantities reported in Customs data are not always reliable. Second, NCVs are obtained from Eurostat, which classifies fuels according to the SIEC. In Annex \complete, we describe in detail our procedures for cleaning firm–fuel–year quantities and for constructing a CN 8-digit–to-SIEC crosswalk.

        We then define the importer-year emission-intensity of fuel imports as
        
        \begin{equation*}
            EI_{jt} = \frac{1}{\sum_{f \in \text{fuels}} x_{jft}} \sum_{f \in \text{fuels}} q_{jft} \times NCV_{ft} \times EF_{ft} 
        \end{equation*}

        \noindent where $x_{jft}$ is the euro amount firm $j$ imported of fuel $f$ in year $t$. The emission-intensity-weighted variant of our fuel consumption is then defined as

        \begin{equation}
            fuel\_consumption^{EI}_{it} :=  \sum_{j \in \text{fuel importers}} x_{ijt} \times EI_{jt}
        \end{equation}

        \item[2.] \textbf{Fuels used in GHG-emitting activities in stationary installations.} Variants \textbf{2a.}, \textbf{2b.} and \textbf{2c.} are defined the same way as (\ref{eq: fuel consumption}), but the set of fuel importers is different. In each of them, we restrict our definition of fossil fuels in Customs and reclassify firms as fuel importers accordingly.

        \vspace{0.3cm}
        
        \begin{itemize}

            \item[2a.] \textbf{Following Energy Balances.} From Energy Balances, we obtain the list of fossil fuels used in Belgium in GHG-emitting activities: \textit{Final consumption - energy use; Transformation input - energy use; and Energy sector - energy use}. We identify the fuels consumed by at least one of these three categories in at least one year between 2005 and 2022, and redefine fuel importers as firms that import positive amounts of one of the fuels identified by energy balances as used in Belgium for GHG-emitting activities.

            \vspace{0.3cm}

            \item[2b.] \textbf{Sector-specific fuels.} Energy Balances report fuel consumption in GHG-emitting activities, broken down by fuel type and economic activity. Examples of such categories include \textit{Final consumption – industry sector – iron and steel – energy use}, \textit{Transformation input – coke ovens – energy use, and Energy sector – blast furnaces – energy use}. We map each Energy Balance category to a NACE 2-digit sector (the full mapping is provided in Appendix \complete). Using this mapping, we construct buyer-specific sets of fuel importers. Specifically, for each buyer, we use its NACE 2-digit code—obtained from Annual Accounts—to identify the set of fuels used in GHG-emitting activities by that sector, and we restrict attention to firms that import those fuels.

            \vspace{0.3cm}

            \item[2c.] \textbf{Following GPT-5.2.} We ask GPT-5.2, the model that powered ChatGPT in February 2026, to identify which CN 8-digit codes from chapter 27 of the HS refer to fossil fuels used in GHG-emitting activities in industrial stationary installations. We then restrict our fossil fuel definition to this subset of fuels and reclassify fuel importers as the set of firm-year observations that import any positive amount of this restricted list of fuels. Appendix \complete provides the list of all goods included in chapter 27 of the HS and identifies the ones flagged by GPT-5.2. We also report the specific prompt we used in Appendix \complete. 

        \end{itemize}
            
        \vspace{0.3cm}
        
        \item[3.] \textbf{Importers that are likely to be fuel suppliers.} We use importers’ NACE 2-digit codes to classify them according to their likelihood of being fossil fuel suppliers. We define three categories: \textbf{highly likely}, \textbf{medium likely}, and \textbf{not likely}. Appendix \complete reports the classification of NACE 2-digit sectors into these three groups. Using this classification, we define two variants of the fuel consumption proxy:

        \begin{align*}
        fuel\_consumption_{it}^{high}     &:= \sum_{j \in \text{high}} x_{ijt} \\
        fuel\_consumption_{it}^{high\_med} &:= \sum_{j \in \text{high} \cup \text{medium}} x_{ijt}
        \end{align*}

        \noindent that is, $fuel\_consumption_{it}^{high}$ restricts the set of fuel importers to firms whose NACE 2-digit sector indicates a high likelihood of supplying fossil fuels, while $fuel\_consumption_{it}^{high}$ additionally includes firms whose sector indicates a medium likelihood.

        \vspace{0.3cm}

        \item[4.] \textbf{Exclude EU ETS-regulated fuel importers.} Because high-emitting fuel importers are more likely to use imported fuels in their own production rather than sell them downstream, we construct a final classification of fuel importers that excludes firms covered by the EU ETS:

        \begin{equation}
            fuel\_consumption_{it}^{non-EUETS} := \sum_{j \in \text{fuel importers}/\text{EUETS}} x_{ijt}
        \end{equation}
    \end{itemize}

    In Table \complete we present the number of firms classified as fuel importers and the euros amount (expressed in 2005 values) imported of fuels in 2005, 2010, 2015, and 2020 for each of the variants. 





\section{Suggestive Evidence of Proxy's Information Content}

In this section, we describe the training sample and provide suggestive evidence on the information content of the proxy.

\noindent \textbf{Training Sample.} The training sample consists of an unbalanced panel of firms for which emissions are observed. These firms fall into two groups. First, EU ETS–regulated firms, for which annual emissions are publicly available from the EU Transaction Log (EUTL). Second, non–EU ETS firms operating in sectors (NACE 19) and (NACE 24). In these sectors, EU ETS coverage is effectively complete, so that all emissions are reported by ETS-regulated installations and non-EU ETS firms are assumed to have zero emissions. Table \complete in Appendix \complete reports EU ETS emissions coverage by sector for 2024 and 2025.\footnote{While we cannot guarantee that EU ETS coverage is exactly 100\% in these sectors in earlier years, the number of ETS-regulated installations is nearly constant over time, suggesting no meaningful changes in coverage.}

Table \complete provides summary statistics of the training sample. Below, we use the sample to compute some descriptive statistics that provide suggestive evidence of the information content of the proxy. In Section \complete, we use the training sample to assess the out-of-sample predictive performance of different models. 


\noindent \textbf{Proxy's Information Content.} One would expect a good proxy for fossil fuel consumption to i) be positively and strongly correlated with firms' emissions, even after controlling for firm size; and ii) to represent a meaningful share of input expenditures for firms that are emission sources, and a negligible share for those that are not. 

We use our training sample to assess both properties. In Table \complete and Figure \complete, we address property i). The table shows the results of three different regression models of emissions into the fuel consumption proxy at the firm-year level. The specifications follow

    \begin{equation}
        \label{eq: reg model}
        emissions_{it} = \eta_i + \gamma_t + \alpha'x_{it} + \beta*fuel\_consumption_{it} + \varepsilon_{it}
    \end{equation}

\noindent where $emissions_{it}$ is firm $i$'s emissions in year $t$, $\eta_i$ is a NACE 2-digit code sector fixed-effect, $\gamma_t$ is a year fixed-effect, $x_{it}$ is a vector of firm-year $it$ covariates, and $fuel\_consumption_{it}$ is the proxy for firms' consumption of fossil fuels. Throughout, the fuel consumption proxy is measured in nominal euros. Emissions are measured in tons of CO$_2$. The specifications differ in the set of covariates $x_{it}$ and the exact definitions of $emissions_{it}$ and $fuel\_consumption_{it}$, as discussed below.

In column (1), we report the results from specification (\ref{eq: reg model}) when $x_{it}$ is empty, the fuel consumption proxy includes purchases from all importers (regulated and non-regulated by the EUETS), and both fuel consumption and emissions are included in levels. The coefficient $\beta$ is positive and significant (t-stat 15.95), although small in magnitude. This is mainly due to the fact that fuel consumption and emissions are both extremely skewed variables with large differences in scale across firms. In levels, the regression interprets the coefficient as the absolute change in emissions associated with a one-unit (one euro) change in fuel spending. Since a one-euro increase in fuel purchasing is negligible relative to the scale of firm-level emissions (with many firms emitting tens of thousands of tons annually), the slope in levels mechanically appears tiny—even though the underlying relationship may be strong.

    In contrast, the log–log specification in column (2) rescales both variables so that the model captures proportional relationships rather than absolute changes. A one-percent increase in fuel spending is economically meaningful for firms of all sizes, and thus the estimated elasticity is large and stable. This point is made explicitly in Figure \ref{fig: emissions vs fuel}, which plots emissions vs fuels across firm-year observations in levels (panel (a)) and in logs (panel (b)).

    Column (2) defines the fuel consumption proxy according to (\ref{eq: fuel consumption}) and (\ref{eq: fuel consumption for EUETS importers}). In column (3), we exclude purchases from EUETS importers as in (\ref{eq: fuel consumption excl euets importers}). This increases the magnitude of the coefficient on fuel consumption as well as the R2 measures, suggesting that indeed purchases from EUETS importers mainly add noise to the proxy.

    Columns (2) and (3) establish that fuel consumption is positively correlated with emissions. However, given that the data set only includes pollutant firms and that emissions are proportional to firms' size, we would likely find a positive relationship between emissions and any firm-level measure increasing in size. In columns (4) and (5), we include firms' log revenue in the set of covariates and find that the coefficient in fuel consumption remains statistically significant (t-stat 6.37 and 7.98, respectively). Thus, our proxy for fuel consumption is correlated with emissions even after controlling for firms' size.

    \begin{table}
        \centering

        \caption{Regression results}
        
        \label{tab: regression}
        
        %\scalebox{0.85}{
          \begin{tabular}{lccccc}
\toprule
& \multicolumn{1}{c}{Levels} & \multicolumn{4}{c}{Logs} \\
\cmidrule(l{3pt}r{3pt}){2-2} \cmidrule(l{3pt}r{3pt}){3-6}
& (1) & (2) & (3) & (4) & (5)\\
\hline
Revenue &  &  &  & 0.509 & 0.482  \\
&  &  &  & (0.021) &  (0.021)  \\
Fuel consumption & 0.000 & 0.403 & 0.428 & 0.112 & 0.144 \\
& (0.000) & (0.014) & (0.014) & (0.018) & (0.018) \\
\hline \hline
Excl. EUETS importers & N & N & Y & N & Y \\
Sector FE & Y & Y & Y & Y & Y\\
Year FE & Y & Y & Y & Y & Y \\
R2 & 0.364 & 0.674 & 0.684 & 0.734 & 0.736 \\
R2 Adj. & 0.345 & 0.664 & 0.675 & 0.726 & 0.728 \\
N & 2689 & 2689 & 2689 & 2689 & 2689 \\
\bottomrule
\end{tabular}
        %}
        \caption*{\footnotesize \textit{Notes}: Table reports estimates of version of specification \ref{tab: regression}. Column (1) reports estimates of regression where both $emissions_{it}$ and $fuel\_consumption_{it}$ are included in levels. Column (2) reports estimates of the log-log specification. Column (3) reports estimates of the log-log specification when $fuel\_consumption_{it}$ excludes purchases from EUETS fuel importers. Column (4) reports estimates of the log-log specification adding $\log(revenue)$ as a control. Column (5) reports estimates of the log-log specification when adding $\log(revenue)$ as a control, and $fuel\_consumption_{it}$ excludes purchases from EUETS fuel importers. Observations are at the firm $\times$ year level. All specifications include NACE 5-digit sector and year fixed effects. Data is an unbalanced panel of EUETS-regulated Belgian firms between 2005 and 2022. It excludes firms for which NACE sectors are singletons within the sample. It only includes firm-years for which emissions are positive.}
    \end{table}

\begin{figure}[ht!]
    \centering
    \caption{Emissions vs Fuel Consumption}
    \label{fig: emissions vs fuel}

    \begin{subfigure}[t]{0.48\textwidth}
        \centering
        \includegraphics[width=\textwidth]{dec25_version/figures/emissions_fuel_levels.png}
        \caption{Levels}
        \label{fig: emissions vs fuels in levels}
    \end{subfigure}
    \hfill
    \begin{subfigure}[t]{0.48\textwidth}
        \centering
        \includegraphics[width=\textwidth]{dec25_version/figures/emissions_fuel_logs.png}
        \caption{Logs}
        \label{fig: emissions vs fuels in logs}
    \end{subfigure}
        \caption*{\footnotesize \textit{Notes}: Figure plots scatterplots of emissions vs the proxy for fuel consumption that excludes purchases from EUETS fuel importers. Panel (A) plots the scatterplot when both emissions and fuel consumption are in levels. Panel (B) reports the scatterplot when both quantities are in logs. Panel (B) also reports the fitted regression line from the specification $
        \log(emissions)_{it} = \alpha + \beta \log(fuel\_consumption)_{it} + \varepsilon_{it}$. Each observation is a firm $
        \times$ year. Data is an unbalanced panel of EUETS-regulated Belgian firms between 2005 and 2022. It excludes firms for which NACE sectors are singletons within the sample. It only includes firm-years for which emissions are positive.}
\end{figure}

We then ask whether the fuel consumption proxy is a meaningful share of input expenditures for firms that are emission sources, and a negligible share for those that are not. Figure \complete shows the kernel density of the input cost share of fuel imports across firms by emitters and non-emitters, separately for sectors NACE 19 and NACE 24. We use the benchmark definition of the fuel consumption proxy, as in (\ref{eq: fuel consumption}). 

\section{Model Selection} \label{sec:model}

    % In this section, we evaluate the out-of-sample performance of different models in predicting firm-year-level emissions. For each model, we conduct a Leave-One-Out Cross-Validation (LOOCV) using firms for which we obseve yearly emissions and compute the nRMSE. The exercise amounts to doing repeated out-of-sample predictions. The idea is to randomly pick a firm among the ones for which we observe emissions, exclude it from the training data, fit the model using the remaining firms, use the fitted model to predict yearly emissions for all years for the firm that was initially excluded, and then use the fact that we observe this firm's emissions to assess how far the inference based on our model is to the actual observed value. See Algorithm \ref{algorithm: loocv} for a detailed description.
    
    % We start off with following model as benchmark:

    % log(emissions_it) = log(revenue_it) + year FE + sector FE

    % We introduce four ingredients to model (\ref{}) and assess the change in out-of-sample predictive performance for each of them in turn. They are:

    % i. flexiblity over the level at which we define the sector fixed effects.
    % ii. introduce the fuel consumption proxy.
    % iii. a intermediary to identify pollutants.
    % iv. calibrate the firm-level predictions to match sector-level known aggregates.

    % Discuss why sector-level at a granular level (5-digit) is useful but problematic.

    % Show the following:
        % Distribution of sector support in the training sample: how many sector FE are estimated with very few firm-years

        % "Unrestricted sector fixed effects may be estimated with substantial noise"?

        % “What fraction of firms (or emissions) would rely on sector FE with weak or no support at deployment?”

    % Introduce partial pooling: formal and intuition.

    % Discuss results

    % Introduce all the tweaks to the fuel consumption proxy.

    % Explain we ran the model for all possible combinations of the tweaks and we chose the combination that provided the best nRMSE within the LOOCV.

    % Discuss the results (what is the chosen proxy? what's the number of surviving obs? is this the proxy that also delivers the best nRMSE after steps 1,3?)

    % Introduce step 1. Discuss results. What's the threshold we chose? Is is based on the same fuel consumption proxy as the one chosen in the step above?

    % Introduce step 3. Discuss results. 

    % Introduce LOSOCV. Discuss results. 

    % Table with four columns: nRMSE, MAPD (E_i > 0), FPR (TPR?), Spearman rho

    % And 6 rows: Benchmark, Partial pooling, Fuel consumption proxy, Weight by prob(E_i > 0), Match sectoral aggregates, LOSOCV
    


    

    
    

%\include{winter26_version/section/conclusion}

\bibliographystyle{apalike} 
\bibliography{refs}

\end{document}