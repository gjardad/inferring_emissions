\documentclass[12pt]{article}

\usepackage[margin=0.75in]{geometry}

\usepackage{comment}
\usepackage{mathpazo}
\usepackage{graphicx}
\usepackage{float}
\usepackage{booktabs}
\usepackage{natbib}
\bibliographystyle{apalike}

\usepackage{fancyhdr}

\usepackage{amsmath}
\usepackage{amsthm}
\newtheorem{theorem}{Theorem}
\newtheorem{corollary}{Corollary}
\newtheorem{lemma}{Lemma}
\newtheorem{proposition}{Proposition}

\pagestyle{fancy}
\fancyhf{}
\rhead{}
\lhead{Gabriel Jardanovski}
\rfoot{\thepage}

\usepackage{hyperref}
\hypersetup{
    colorlinks=true,
    linkcolor={blue},
    citecolor={blue},
    }

\title{\textbf{Carbon Policy in a Networked Economy}}
\date{}

\begin{document}
    %\maketitle
    \sloppy
    \thispagestyle{fancy}

    \section*{Notes on TROPOMI data}

    \noindent \textbf{TROPOMI.} Air pollutants absorb sunlight at specific wavelengths that are largely invisible to the human eye. For example, nitrogen dioxide (NO$_2$) absorbs light in the near-ultraviolet and blue visible parts of the spectrum, while methane (CH$_4$) absorbs light in the short-wavelength infrared.
    
    To measure the concentration of these gases in the air, one needs a spectrometer that records sunlight reflected by the Earth’s surface and atmosphere over a wide range of wavelengths - from ultraviolet to infrared. This is precisely what the TROPOspheric Monitoring Instrument (TROPOMI) does.

    TROPOMI is a spectrometer carried by the Sentinel-5 Precursor (Sentinel-5P), a satellite developed by the European Space Agency (ESA) and launched in 2017. As Sentinel-5P orbits the Earth, TROPOMI measures the spectral composition of reflected sunlight over rectangular areas of approximately 3.5 km × 5.5 km, referred to as pixels. Because atmospheric gases have known and distinct absorption features, TROPOMI can retrieve gas concentrations for each pixel: it measures how strongly reflected sunlight is attenuated across wavelengths, and greater attenuation of particular wavelengths implies a higher concentration of the absorbing gas within that pixel.

    \vspace{0.5cm}

    \noindent \textbf{NO$_x$.}% paragraph about why NO_2 is relevant for us

    \vspace{0.5cm}

    \noindent \textbf{Data.} Sentinel-5P completes an orbit approximately every 100 minutes. During each orbit, TROPOMI measures atmospheric gas concentrations for thousands of pixels along a wide swath of the Earth’s surface. For any given orbit, only a small subset of these pixels lies within Belgium. Over time, both the angle at which TROPOMI observes the ground and the angle at which the Sun illuminates it change. This causes the exact locations of pixels to vary slightly across orbits. By stacking pixels across many orbits, we recover a picture of atmospheric gas concentrations across the entire country.

    We collect data from orbits between May and September 2019. We chose the summer period because nitrogen dioxide has a shorter photochemical lifetime during summer months [which strengthens the spatial link between observed NO$_2$ concentrations and nearby emission sources.] We focus on 2019 because it is the earliest year for which TROPOMI provides measurements at a spatial resolution of approximately 3.5 km × 5.5 km; during the summer of 2018, measurements were still taken at a coarser resolution. Between May and September 2019, there are 413 satellite overpasses containing pixels that intersect Belgium. Due to data-processing constraints, we randomly select 138 of these overpasses.\footnote{Each orbit-level data file is large, and downloading the 138 selected orbits required more than 24 hours. We chose 138 because it is a third of 413 (rounded up).}

    We stack the resulting pixel-level observations and aggregate them onto a regular spatial grid with a resolution of $0.02 \times 0.02$. 
    %Although individual TROPOMI pixels cover areas of several square kilometers, successive satellite overpasses observe the same locations from slightly different angles, leading to overlapping pixel footprints. By stacking observations across many overpasses and aggregating them onto a finer regular grid, we exploit this overlap to localize persistent NO₂ hotspots at a spatial scale finer than the native pixel size. 

\end{document}
